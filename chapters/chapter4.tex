\section{MIDIS用户操作流程}

\maintopics{
    \myitem{MIDIS操作流程}
    \myitem{MIDIS在使用过程中常见问题}
    \myitem{MIDIS打包操作}
}

\subsection{MIDIS具体操作流程}

\mycontent{首先启动MIDIS,用户登录。用户名分别为:cst,hfss,root,密码为iei813。}

\myfig{0.4}{use_1}{MIDIS用户登录界面}

\mycontent{MIDIS的相关操作都需要依赖数据库,所以这里需要启动数据库范围,如图5-2所示 工具 -- 启动数据库。}

\myfig{0.4}{use_0}{启动数据库}

\newpage

\mycontent{然后进入主界面,新建工程,这里需要选择工程的位置和设置工程的名字。如果需要打开已经创建的工程,则选择菜单栏:工程 ---> 打开工程。然后选择.project文件。具体操作如图5-3所示。}

%{\color{red}\textbf{ll}}


\myfig{0.9}{use_2}{创建工程和打开工程}


\mycontent{在新建的工程下添加模型文件到工程中,具体操作如图5-4所示。这里可以分别选择HFSS文件和CST文件。\color{red}\textbf{这里需要注意的是如果是添加CST文件,则需要将该CST文件通过CST软件打开,这时会产生一个CST工程目录文件,工程目录中的文件才是我们需要解析的。}}


\myfig{0.9}{use_3}{添加模型文件}


\mycontent{添加的模型需要进行解析操作,最新的版本在导入模型后可以自动解析,如果不能自动解析,可以通过图5-5的右键操作进行手动解析。解析成功后,信息窗口会有解析成功提示。}

\myfig{0.9}{use_4}{解析模型}

\newpage

\mycontent{接下来需要对解析后的模型进行算法配置,{\color{red}\textbf{这里需要注意的是只要解析后的模型才能进行算法配置,此时解析后的模型需要被选中,如果发现无法选择算法,需要检查模型是否解析,模型是否选中。}}然后进行图5-6所示的操作,这里常用的两个算法有For算法,可以对变量空间进行遍历。遗传算法用来进行全局优化。}

\myfig{0.9}{use_5}{选择算法}

\newpage

\mycontent{以遗传算法为例,算法配置面板如图5-7所示,这里可以按照需求修改每代种群的个数和遗传代数,其他参数按照默认值即可。}

\myfig{0.9}{use_6}{遗传算法配置}

\mycontent{然后进行变量范围设置,点击 + 添加模型中的变量,编码长度控制了变量优化时的精度,具体的公式为:$\frac{max-min}{2^n-1}$,其中n为编码长度,max,min分别为变量最大值和最小值。一般情况下,需要勾选参数制约来设置参数之间的大小关系。{\color{red}\textbf{当添加多个变量时,需要按照HFSS中的模型变量顺序依次进行添加,切不可错乱顺序。}}}

\myfig{0.9}{use_7}{变量设置}

\mycontent{参数制约需要在 restrict.ini 中进行编写。第一个制约条件为 [Res1] restrict = ...。第二个为 [Res2] restrict = ...。变量用美元符 \$ 表示,不能使用各种括号。示意图如图5-9所示。}

\myfig{0.9}{use_8}{参数制约}

\newpage

\mycontent{数据库的配置如图5-10和图5-11所示。这里面有五个默认的数据库分别为: HFSSDataBase,Project,CST,Research,Graduate 根据需求选择数据库。用户名和密码,这里已经有了默认值,不需要改写。显示集合中,会使用模型的名称作为数据库中集合的名称,点击确认集合按钮后,如果数据库中存在该集合,则会提示,你可以修改集合名称创建一个新集合,也可以继续使用这个集合名称。相关变量配置可以设置模型的一些标签和厚度等信息。}

\myfig{0.9}{use_9_1}{选择数据库配置}

\myfig{0.9}{use_9_2}{数据库配置}

\newpage

\mycontent{适应度的配置过程如图5-12、5-13、5-14所示。这里需要根据实际的设计需求选择不同的适应度插件,对于单个HFSS模型,可以选择 第一项 Fitness\_Simple\_HFSS 插件,如图5-13所示。{\color{red}\textbf{这里需要注意,起始频率和终止频率必须在HFSS模型中设置的扫频范围内。}}。如果需要对多个模型分别设置其适应度,可以选择最后一个插件 Fitness\_Multiple\_Phase 这里可以选择需要计算的S参数类型,阈值比较,相位范围等。}

\myfig{0.9}{use_12}{适应度配置1}

\myfig{0.9}{use_10}{适应度配置2}

\newpage

\myfig{0.9}{use_11}{适应度配置3}

\mycontent{所有设置完成后,需要将工程进行保存,具体操作如图5-15所示。}

\myfig{0.5}{use_13_0}{保存工程}

\newpage

\mycontent{打开任务管理面板,添加任务,输入需要的计算节点数目。然后点击 开始 按钮即可启动任务。}

\myfig{0.9}{use_13}{在任务管理面板中启动任务}

\newpage

\subsection{MIDIS操作简化流程}

\mycontent{第一小节具体介绍了操作的流程和需要注意的事项(红色加粗部分),可以将以上内容简化为图5-17所示的流程图。}

\myfig{0.45}{4-2}{MIDIS操作流程图}

\newpage

\subsection{MIDIS源代码打包}

\mycontent{针对Python的打包工具主要有三种,具体区别如图5-18所示。}

\myfig{0.6}{4-1}{Python打包工具对比}

\mycontent{这里我们使用PyInstaller进行MIDIS的打包,首先按照python2.7 版本的 PyInstaller工具包,然后在命令行窗口定位到MIDIS的主py文件目录,这里需要注意,整个路径不能有中文字符,然后执行下面代码即可。}

\begin{lstlisting}
pyinstaller -F --icon=.\\MIDIS.ico -c MIDIS.py
\end{lstlisting}

\mycontent{其中 第一项 \-F 指定打包的exe文件的图标,-c MIDIS.py 需要打包的主py文件。成功打包后,会在MIDIS目录下产生两个文件夹,分别为 build 和 dist 其中 打包的exe文件位于dist文件夹中。最后需要将exe文件和插件代码以及一些图片,外部文本文件拷贝在一起形成完整的打包软件。最终效果图如图5-19所示。}

\myfig{0.8}{4-3}{MIDIS打包效果图}

\mycontent{使用PyInstaller在MIDIS打包过程中会遇到一些问题,就本人在打包过程中遇到的一些问题与解决方案给出以下说明。}

\begin{enumerate}[itemindent=1em] 
\item \mycontent{{\color{red}\textbf{ascii / utf8 codec can't encode character ... in position ...}}}

\mycontent{这种情况一般是因为编码问题,首先检查源码文件中是否有中文路径。然后对应着错误提示
找到出错的文件,这里我遇到的是ntpath.py 的编码问题,
第一步 在 python下的lib下的ntpath.py中添加}
\begin{lstlisting}
#coding=utf-8
import sys   
reload(sys) 
sys.setdefaultencoding('utf-8')	
\end{lstlisting}
\mycontent{第二步:
将约84行的result\_path = result\_path+ p\_path替换为result\_path = str(result\_path) + str(p\_path)。}


\item \mycontent{{\color{red}\textbf{no module named ...}}}

\mycontent{很明显,在编译过程中找不到需要的库,但是我们在Python环境中是可以运行的,这些库也已经安装过。
这里可能有一个原因,在安装Python库的过程中,有两种安装方式,一种是将库的源码安装到site-packages中,第二种方式是将库的源文件
进行编译打包后安装到目录中(egg文件)。第二种方式PyInstaller是无法解析打包文件中的库,会报错。
}

\mycontent{首先我们在Python目录中找到报错的库,确认库存在,如果是只有egg文件,且没有其他同名目录文件,
则需要下载源代码重新安装到目录中,或者直接将同名目录代码拷贝到 site-packages 中去。
}

\item \mycontent{{\color{red}\textbf{No module named ... 找不到库的依赖文件}}}

\mycontent{和上一个问题不同的是,它并不缺少库,而是找不到库的一个依赖文件,但是经过检查
库的源文件和依赖文件也在。这时可以采用的解决方案是将该包拷贝到程序源代码目录下
,然后在代码主py文件中强制将该包导入。
在安装过程中我遇到的问题是 No module named FxiTk,并且在编译日志中看到
WARNING : Removing import "FixTk"
FxiTk.py是Tkinter库的依赖文件。将该文件所在的目录(lib-tk)拷贝到代码目录中,然后在程序主Py文件(MIDIS.py)
添加一行代码: sys.path.append('lib-tk')}

\item \mycontent{{\color{red}\textbf{运行 exe 文件 卡死}}}

\mycontent{请检查中文路径,将文件放在全英文环境中,然后重新运行。}

\item \mycontent{{\color{red}\textbf{缺少 DLL 或其他与Windows环境相关的错误}}}

\mycontent{这可能和Windows环境有关,检查 microsoft visual c++ 的版本,如果版本过高,请降至
microsoft visual c++ 2010。}

\end{enumerate}

\vspace{0.5cm}

\mycontent{以上就是我遇到的一些常见问题,当然在具体操作过程中也许还会遇到一些其他问题,因为
这个我没有遇到过,所有无法给出解决方案。当遇到问题时,希望能够冷静等待,不要轻易放弃,追根溯源,一定
会找到一个解决办法。}


