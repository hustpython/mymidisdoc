\section{MIDIS主体框架}

\maintopics{
    \myitem{MIDIS的主体框架}
    \myitem{Pydocview代码分析}
    \myitem{服务插件的编写}
}

\subsection{wxPydocview框架}

\mycontent{
    MIDIS的GUI框架是在pydocview框架进行扩展。pydocview是wx库中自带的一个GUI框架,通过继承的方式进行自定义。
功能函数主要包括,主函数DocApp,主界面函数DocTabbedParentFrame,和文档管理函数DocManager(docview.py文件中)以及其他功能函数构成。}
\myfig{0.7}{3}{主体框架}

\mycontent{
    如图1-1所示,每一个MIDIS中每一个区域都有相关的服务文件控制,其中菜单栏中的服务插件是我们关注的重点。其他区域的变动性不是很大,首先看一下菜单栏的相关内容。
}
\mycodelist{TabbedParentFrame.py 代码摘录}
%%%%%%%%%%%%%%%%%%%
\setstretch{1.2} 

\begin{lstlisting}
class TabbedParentFrame(pydocview.DocTabbedParentFrame):
    def CreateDefaultMenuBar(self, sdi=False):
        menuBar = wx.MenuBar()
        fileMenu=wx.Menu()
        fileMenu.Append(wx.ID_NEW,_(u"&新建文件\tCtrl+N"),_(u"新建文件"))
        fileMenu.Append(wx.ID_OPEN,_(u"&打开…\tCtrl+O"),_(u"打开文档"))
        fileMenu.Append(wx.ID_CLOSE,_(u"&关闭"),_(u"关闭文件"))
        fileMenu.Append(wx.ID_CLOSE_ALL,_(u"&关闭所有"),_(u"关闭所有文件"))
        menuBar.Append(fileMenu,_(u"&文件"))
        ......
        editMenu = wx.Menu()
        editMenu.Append(wx.ID_UNDO, _(u"&撤销\tCtrl+Z"), _("Reverses the last action"))
        wx.EVT_MENU(self, wx.ID_UNDO, self.ProcessEvent)
        wx.EVT_UPDATE_UI(self, wx.ID_UNDO, self.ProcessUpdateUIEvent)
        editMenu.Append(wx.ID_REDO, _(u"&恢复\tCtrl+Y"), _("Reverses the last undo"))
        wx.EVT_MENU(self, wx.ID_REDO, self.ProcessEvent)
        menuBar.Append(editMenu, _(u"&编辑"))
        menuBar.Append(viewMenu, _(u"&查看"))
        menuBar.Append(helpMenu, _(u"&帮助"))
        return menuBar
    def CreateDefaultToolBar(self):
        self._toolBar = self.CreateToolBar(wx.TB_HORIZONTAL | wx.NO_BORDER | wx.TB_FLAT)
        self._toolBar.AddSimpleTool(wx.ID_NEW, New.GetBitmap(), _("New"), _("Creates a new document"))
        ......
        return self._toolBar
    def OnAbout(self,event):
        dialog=AboutDialog.AboutDialog()
        dialog.CenterOnParent()
        if dialog.ShowModal()==wx.ID_OK:
            dialog.Destroy()
\end{lstlisting}

\mycontent{
    首先可以看出TabbedParentFrame 类是继承自pydocview中的DocTabbedParentFrame中的,在pydocview中打开这个类,可以看出类中实现了相同的函数CreateDefaultMenuBar和CreateDefaultToolBar,只不过原函数中的相关字符串都是英文,这里我们将其自定义为中文。TabbedParentFrame类在菜单栏中实现了4个默认的选项卡即文件,编辑,查看,帮助。在大部分软件中,这四个选项卡都会存在,而且他自定义的选项卡会通过Service类进行添加,后面会具体分析到。
回到 Entry.py 文件,
}
\begin{lstlisting}
    embeddedWindows = pydocview.EMBEDDED_WINDOW_TOPLEFT pydocview.EMBEDDED_WINDOW_BOTTOM|pydocview.EMBEDDED_WINDOW_BOTTOMLEFT
\end{lstlisting}
\mycontent{设置了界面的布局分三个部分}
\begin{lstlisting}
    projectservice = self.InstallService(ProjectEditor.ProjectService(_(u"工程"),pydocview.EMBEDDED_WINDOW_TOPLEFT))
    outlineservice = self.InstallService(OutlineService.OutlineService("Outline", pydocview.EMBEDDED_WINDOW_BOTTOMLEFT))
    self.messageservice = self.InstallService(MessageService.MessageService(_(u"信息"),pydocview.EMBEDDED_WINDOW_BOTTOM))          
\end{lstlisting}
\myfig{0.9}{1}{MIDIS分区图}
\mycontent{
初始指定了默认选项卡的位置,工程选项卡在左上侧,Outlineservice在左下侧(空白处,好像没有用到),日志输出在右侧。
还有文档查看,这个是通过docview的docManager函数和STCTextEditor进行管理。在MIDIS中的具体效果如图1-2所示。
对MIDIS普通的扩展主要是对其增加服务插件,Pydocview.py 中 class DocApp(wx.App):是整个GUI的入口,在Entry.py函数中class MainApp(pydocview.\\DocApp):
进行了重新的自定义包括设置APP的名称,GUI的布局,菜单栏,状态栏,工程目录等。
InstallService函数,将各种服务添加到全局变\_service(继承至pydocview)中至此各种服务插件都加载进去了,通过界面中的相关操作,调用服务插件中的各种函数即可实现相应功能。
那么MIDIS是将我们写的选项卡添加到相应的下拉框中去的呢?
我们以算法服务为例:找到:AlgorithmService.py文件如果你需要将一个服务类添加到GUI中,则InstallControls是一个必须实现的函数,并且需要在Entry.py中导入该类和实例化。接下来需要弄明白两个事情,InstallControls实现了什么功能以及如何在Entry将其实例化。
}

\subsection{服务插件的编写}

\myfig{0.4}{2}{算法服务实例菜单}

\mycontent{如图1-3所示是算法服务插件的四级菜单效果图}
\mycodelist{算法服务中的InstallControls函数实现}
\setstretch{1.2} 
\begin{lstlisting}
def InstallControls(self,frame,menuBar = None, toolBar = None, statusBar = None, document = None):
    projectMenuPos = menuBar.FindMenu(_(u"工程"))
    if projectMenuPos == wx.NOT_FOUND:
        projectMenuPos = 0
    if menuBar.FindMenu(_(u"优化")) == wx.NOT_FOUND:
        optimizeMenu = wx.Menu()
        menuBar.Insert(projectMenuPos+1,optimizeMenu,_(u"优化"))
    optimizeMenu = menuBar.GetMenu(menuBar.FindMenu(_(u"优化")))
    algSelectMenu = wx.Menu()
    optimizeMenu.AppendSubMenu(algSelectMenu,_(u"选择算法"))
    self.AlgSelectId = optimizeMenu.FindItem(_(u"选择算法"))
    wx.EVT_UPDATE_UI(frame,self.AlgSelectId,self.ProcessUpdateUIEvent)
    algSelectMenu = optimizeMenu.FindItemById(self.AlgSelectId).GetSubMenu()
    algorithmPluginPath = os.path.join(sysutil.mainModuleDir,"Plugin","algorithmplugin")
    i = 0
    for dirpath,dirnames,filenames in os.walk(algorithmPluginPath):
        i += 1
        if i == 1:
            for dirname in dirnames:
                if Debug:print dirname
                subMenu = wx.Menu()
                algSelectMenu.AppendSubMenu(subMenu,dirname)
        else:
            if dirpath not in sys.path:
                sys.path.insert(0,dirpath)
            for filename in filenames:
                if filename != "__init__.py" and (os.path.splitext(filename)[1] == ".py" or 
                                                    os.path.splitext(filename)[1] == ".pyc") and "plugin" in filename: 
                    if os.path.splitext(filename)[0] not in sys.modules:
                        try:
                            moduleName = __import__(os.path.splitext(filename)[0])
                            module = moduleName.InstallPlugin()
                            self._algorithmService.append(module)
                            printMessage(u"插件 %s 加载成功"%filename)
                        except:
                            printMessage(u"插件 %s 加载失败"%filename)
            if dirpath in sys.path:
                del sys.path[0]
    return
\end{lstlisting}

\mycontent{
    我们需要将"算法"选项插入到"优化"选项卡下的二级菜单中,首先通过FindMenu函数确定"优化"选项是否已经建立,如果没有建立,找到"工程"的ID,在其后面通过menuBar.Insert函数创建(3-8)。然后AppendSubMenu函数按服务实例化的先后顺序在优化菜单下创建"选择算法"菜单(在Entry.py中模型解析服务在算法服务之前进行了实例化),在算法插件中,我们按照不同的算法和算法实现的功能建立了多个以算法类型命名的文件夹,每一个文件夹下会有多个算法插件,文件夹名称构成了第三级菜单(16-22),这个菜单下的各种算法插件,插件通过"InstallPlugin"函数,其实是调用了InstallService函数,最终将最后的算法插件插入到最后的菜单中(24-38)。

到最底层的服务插件会需要一个响应函数,用来响应点击该菜单后的效果,因为算法插件最后的实现是在plugin/algorithmplugin/GA/1GAPlugin.py中,在这些文件的InstallControl函数下的wx.\_EVTMENU\\(frame,self.ALGORITHM,frame.ProcessEvent)就是用来实现响应操作,\\self.ALGORITHM表示选择的菜单选项,frame.ProcessEvent是用来处理响应的。

在Entry.py服务的实例化是通过InstallService函数实现,在pydocview中进行定义。
}
\begin{lstlisting}
def InstallService(self, service):
    service.SetDocumentManager(self._docManager)
    self._services.append(service)
    return service
\end{lstlisting}
\mycontent{可以看到这个函数的有一个功能就是将服务添加到全局变量\_services中去。}
\begin{lstlisting}
    self.frame=TabbedParentFrame(docManager,None,-1,self.GetAppName(),embeddedWindows=embeddedWindows,minSize=150)
    self.frame.Show()
\end{lstlisting}

\mycontent{当所有服务通过InstallService进行实例化之后,将默认的菜单embeddedWindows,文档管理docManager,APP名称self.GetAppName()等组合一起显示,通过show()函数就可以将所有的界面相关内容展示出来。那么我们自己自定义的服务是如何显示的呢,这个还是得通过embeddedWindows.py文件,虽然在这个文件当中,我们只是实现了几个默认的菜单选项,再看一下这个类的声明。}

\mycontent{class TabbedParentFrame(pydocview.DocTabbedParentFrame):
这个类继承了pydocview的DocTabbedParentFrame类,因为TabbedParentFrame没有初始化(init)函数,而且在pydocview的DocTabbedParentFrame类中,有初始化函数,所有这个初始化函数被继承到了TabbedParentFrame当中,初始化时调用了\\self.\_InitFrame(embeddedWindows, minSize),具体代码如下:}
\mycodelist{DocTabbedParentFrame类下的\_InitFrame实现}
\begin{lstlisting}
def _InitFrame(self, embeddedWindows, minSize):
    for service in wx.GetApp().GetServices():
        service.InstallControls(self, menuBar = menuBar, toolBar = toolBar, statusBar = statusBar)
        if hasattr(service, "ShowWindow"):
            service.ShowWindow()    
\end{lstlisting}
\mycontent{这个函数遍历\_services中的每一个服务,然后调用服务中的InstallControls,从而实现了菜单栏的加载。
以上了解了服务插件添加到菜单栏的流程,可以简化为图1-4所示流程。}

\myfig{0.4}{4}{MIDIS构建插件流程图}

\subsection{本章小结}

\xiaojie{
    \myitemxiaojie{pydocview是一个支持插件扩展的框架,支持文档管理,插件扩展,文档编辑等功能}
    \myitemxiaojie{pydocview下主要包括pydocview.py,docview.py等文件}
    \myitemxiaojie{服务插件的编写主要是通过InstallControls函数实现}

}