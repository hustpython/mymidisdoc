\section{数据库实现}

\maintopics{
    \myitem{数据库工具使用}
    \myitem{数据库接口实现}
    \myitem{数据分析插件实现}
}

\subsection{数据库接口类与数据库工具使用}

\mycontent{数据库的相关操作在 Source-->tool-->DBclassService.py 文件中。该文件的类主要实现了数据库的 C(创建)、R(查询)、U(更新)、D(删除)等基本操作。
Python是通过pymongodb库与mongodb交互,通过该库可以连接数据库和其他数据库操作。代码清单4-1实现了连接本地数据库 database的操作。
这里需要注意的是在本地计算机和机群中Ip不同,其中 Source--->tool 中
DBclassService.py、DBconfigService.py、DistributedComputerService.py、TaskScheduleService.py 几个文件中都有关于Ip的绑定,所以在本地和机群中使用时请
分别进行设置。即在本地开发时使用本地ip:127.0.0.1、在机群中部署时使用局域网ip,如84号机的Ip为 192.168.1.84。}

\mycodelist{Mongodb数据库的连接}

\begin{lstlisting}
from pymongo import MongoClient
connect=MongoClient('127.0.0.1')
databse = connect['databse']
\end{lstlisting}

\mycontent{在实现数据存储之前,有必要了解一下需要存储哪些键值。一个模型的全部变量和优化变量都是需要的,频率、仿真数据都是列表。仿真数据可以根据不同的情况,设置每一个元素的值,比如是反射率、或幅值和相位两个值。
其他信息,比如模型位置、模型存储位置、时间等可作为辅助信息。}

\mycontent{initialparam函数用来初始化有些固定的变量值,比如变量的名称,模型文件的路径等信息,这些信息在创建任务的时候就已经固定下来,不会改变。}

\mycontent{insertfile是提供给外部的数据存储接口,同时它会调用 insertData2Table 函数。主要看一下 insertData2Table 函数的实现。}

\mycodelist{数据存储函数实现}

\setstretch{1.2} 

\begin{lstlisting}

def insertData2Table(self,dbname,user,pwd,targetfile,Collectname,Fre,Tags,Topos,Thickness,\
                         var_opti_name,temVarM,VarMatrix,Reflect,category):
        '''insert one data to collection'''
        
        self.collectname = Collectname
        db = self.connect[dbname]
        #db.authenticate(user,pwd)
        self.data = db[self.collectname] 
        timeinfo = time.strftime('%Y-%m-%d %H:%M:%S',time.localtime())  
        try:    
            datum = {           
                'name' : Collectname,
                'topology':Topos,
                'tags' :Tags,
                'var_opti_name':var_opti_name,
                'var_opti':temVarM,
                'vars' :VarMatrix ,
                'thickness' :Thickness,
                'performance':'',
                "f":Fre,
                "value":Reflect,
                "category" :category,
                'location':targetfile,    
                'time':timeinfo
            }
            self.data.insert_one(datum)
        except:
            print u'数据存储失败,请检查数据格式'
    
\end{lstlisting}

\mycontent{这里主要是将需要存储的数据以字典的形式保存在变量datum中,然后通过 pymongodb的insert\_one函数将数据存储在数据库中。}

\mycontent{数据库的查询是根据优化变量,优化变量值是区分不同仿真数据的唯一标准,因为优化变量值是以列表的形式存储,导致优化变量在存储时顺序
不同会产生不同的查询结果,这里为了统一标准,在选择算法 ---> 设置变量范围时需要按照HFSS模型中的变量顺序依次添加,不可错乱顺序。数据查询是通过 
find\_one 函数实现。返回值为整个数据的键值信息。}

\mycodelist{数据库查询函数实现}

\setstretch{1.2} 

\begin{lstlisting}
def queryMatrix(self,dbname,user,pwd,Collectname,value):
    try:           
        self.collectname = Collectname          
        db = self.connect[dbname]
        self.data = db[self.collectname]     
        self.query_result = self.data.find_one({'var_opti':value})
    except:
        print u'查询变量失败'
        return
    return self.query_result
\end{lstlisting}

\mycontent{mongodb自带的工具可以很方便地将各种格式的数据文件进行导入、以及导出各种数据形式的文件和备份操作。
使用这些工具可以编写一些脚本命令,进行数据处理。}

\mycontent{首先是将数据库中的数据导出为数据文件,当我们将Mongodb添加到环境变量之后,里面的工具就可以直接使用。}

\mycodelist{数据导出}

\begin{lstlisting}
mongoexport -d dbname -c collectionname -o file --type json/csv -f field
\end{lstlisting}

\mycontent{
    参数说明:\\
    \indent-d :数据库名\\
    \indent-c :collection名\\
    \indent-o :输出的文件名\\
    \indent--type : 输出的格式,默认为json\\
    \indent-f :输出的字段,如果-type为csv,则需要加上-f "字段名"\\
}

\mycodelist{数据导入}

\begin{lstlisting}
mongoimport -d dbname -c collectionname --file filename --headerline --type json/csv -f field   
\end{lstlisting}

\mycontent{
    参数说明:\\
    \indent-d :数据库名\\
    \indent-c :collection名\\
    \indent--type :导入的格式默认json\\
    \indent-f :导入的字段名\\
    \indent--headerline :如果导入的格式是csv,则可以使用第一行的标题作为导入的字段\\
    \indent--file :要导入的文件\\
}

\mycodelist{数据备份}

\begin{lstlisting}
mongodump -h dbhost -d dbname -o dbdirectory
\end{lstlisting}

\mycontent{
    参数说明:\\
    \indent-h: MongDB所在服务器地址,例如:127.0.0.1,当然也可以指定端口号:127.0.0.1:27017\\
    \indent-d: 需要备份的数据库实例,例如:test\\
    \indent-o: 备份的数据存放位置,例如:/home/mongodump/,当然该目录需要提前建立,这个目录里面存放该数据库实例的备份数据。\\
}

\mycodelist{数据恢复}

\begin{lstlisting}
mongorestore -h dbhost -d dbname --dir dbdirectory
\end{lstlisting}

\mycontent{
    参数说名:\\
    \indent-h: MongoDB所在服务器地址\\
    \indent-d: 需要恢复的数据库实例,例如:test,当然这个名称也可以和备份时候的不一样,比如test2\\
    \indent--dir: 备份数据所在位置,例如:/home/mongodump/itcast/\\
    \indent--drop: 恢复的时候,先删除当前数据,然后恢复备份的数据。就是说,恢复后,备份后添加修改的数据都会被删除,慎用!\\
}

\newpage

\subsection{数据库与仿真数据交互}

\mycontent{使用数据库的目的是对仿真数据进行查询和存储,两者的交互操作在 Plugin--->fitnessplugin--->适应度插件中。图4-1展示了两者交互关系。
当算法传入变量参数之后,首先对该变量在数据库中进行记录查询,查询时间一般可以忽略不计,在使用For算法进行变量空间遍历时,若计算过程出现
中断,则可以通过数据库查询快速恢复到之前计算状态即实现断点续算功能。同时在使用遗传算法时,不同
代之间若存在相同的个体也可以直接查询返回,大大提高了计算效率。接下来具体看一下代码实现。}

\myfig{0.45}{3-1}{仿真数据与数据库交互}

\mycodelist{FitnessRun函数数据库查询部分代码}

\setstretch{1.2}

\begin{lstlisting}
def FitnessRun(self,temp_schedulename,objectnumber,varxs,algParam,modeleparam,param_configlist,ReflectService):
    for i in range(objectnumber[1]):
        first_occour = varxs.index(varxs[i])
        Task_unit = task_unit()
        taskinfo_objectnumber=[objectnumber[0],(i+1)]
        countinfo=1
        Task_unit.set_varMatrix(varxs[i])
        if  first_occour == i:
            Task_unit.set_unit_number(i)
            query_result =self.db.queryMatrix(dbname,user,pwd,dbHFSSfilename,varxs[i])
            Task_unit.set_hfssnamedb(dbHFSSfilename)
            Task_unit.set_dbname(dbname)
            Task_unit.set_user(user)
            Task_unit.set_pwd(pwd)
            found = False
            if not query_result:
                found = False
            else:
                found = True
            if found:
                t2=query_result['f']
                self.fre=[float(x) for x in t2]
                countinfo=0
                task_database_info=[self.databasefile,self.tablename,self.projectname,self.taskname,taskinfo_objectnumber,countinfo]
                Task_unit.set_database_info(task_database_info)
                R = query_result['value']
                Fitness = self.countfitness(self.fre, R)
                Task_unit.set_finished()
            else:
                pass
\end{lstlisting}

\mycontent{可以看出当在数据库中查询到记录后,主要提取了模型的频率和S参数值,然后调用计算适应度函数。
数据的存储是在 callback\_receive 函数中,当从远程服务器获取到计算结果文件后,对结果文件进行解析存储。}

\mycodelist{callback\_receive函数 从服务器获取结果文件、进行解析存储}

\setstretch{1.2}

\begin{lstlisting}
def callback_receive(self, server):
    if not server.connect():
        return False
    task = server.get_task()
    task_database=task.get_database_info()
    server_homepath = server.get_homepath()
    server_file_path  = os.path.join(".\\", self.SERVER_DIR, self.projectname, self.taskname,  u'generation'+u'%s'%task_database[4][0]+u'n%s'%task_database[4][1])
    remote_file = os.path.join(server_file_path,os.path.basename(task.get_output_m()))
    print remote_file,'remote_file,receive'
    local_file = task.get_output_m()
    count = 0
    if server.get(remote_file, local_file):
        filepath = os.path.join(server_homepath, self.SERVER_DIR, self.projectname, self.taskname, u'generation'+u'%s'%task_database[4][0]+u'n%s'%task_database[4][1])
        cmd = "powershell cd ~;Remove-Item %s -recurse;quit"%filepath
        server.exec_command(cmd)
        server.set_unused()
        fre,R=get_data_R(local_file)
        if Debug:print fre,R
        fitness_value = self.countfitness(fre, R)
        self.fre=fre
        self.R =R
        task.set_private_data(fitness_value,R)
        task_database=task.get_database_info()
        task_varmtrix=task.get_varMatrix()
        objectnumber=u'第%s代'%task_database[4][0]+u'第%s个个体'%task_database[4][1]
        hfsspaths =task.get_newfileabs()
        dbname = task.get_dbname()
        user = task.get_user()
        pwd = task.get_pwd()
        task.get_newfileabs()
        dbhfss =task.get_hfssnamedb()
        var_opti_name =task.get_var_opti_name()
        self.DATAbase.insertfile(dbname,self.category,user,pwd,hfsspaths,var_opti_name,dbhfss,self.fre,task_varmtrix,R)
        printMessage(u'%s'%objectnumber+u'计算完成')
\end{lstlisting}

\subsection{基于数据库的数据分析插件}
\mycontent{在数据库的基础上,可以很方便地设计数据分析插件,对数据进行展示,提取。在设计
数据分析插件的界面时,有几个功能要考虑到,数据库/集合的选择、变量解析与选择、变量之间的关系操作、适应度配置、数据导出、数据展示。如图4-2所示是MIDIS中
解空间分析的界面。}

\myfig{0.8}{3-2}{解空间分析配置图}

\newpage

\mycontent{如图4-3所示,是数据分析插件的开发流程图,首先是要设计一个操作界面。通过数据库的操作获取到需要的数据,然后设计数据处理算法对数据进行处理,最后对处理的数据进行绘图展示和数据文件保存。}

\myfig{0.5}{3-3}{数据分析插件开发流程}

\mycontent{MIDIS中的数据分析插件在 Plugin--->dataprocess中,这里我们以 DataProcessplugin.py
解空间分析插件为例看一下其中降维算法的实现。
首先templist列表存储了选择的集合的所有数据(2),所有数据的频率都相同,可以用self.Frequency表示(2),变量名称也是
相同的用varnamelist表示(5)。关键点在于如何将多维的变量的组合与适应度的关系降维至二维,其实这个很简单,我们选择我们需要的两个变量
其他变量的排列组合对应多个适应度(6-12),我们只选取其中适应度最大的一个作为这两个变量的值,这里我们使用一个字典进行存储,当遇到相同的键值时通过比较适应度大小来更新(14-16)。}

\mycodelist{解空间降维算法实现}

\setstretch{1.2}

\begin{lstlisting}
def DataFromDatabase(self):
    templist =list(self.get_data[self.gridview.GetCellValue(0,0)].find())
    self.Frequency =templist[0]['f']
    DataDict = dict()
    varnamelist  = templist[0]['var_opti_name']
    for RefInf in templist:
        x_index = varnamelist.index(self.gridx.GetCellValue(0,0))
        x_d = str((RefInf['var_opti'])[x_index])
        y_index = varnamelist.index(self.gridy.GetCellValue(0,0))
        y_d = str((RefInf['var_opti'])[y_index])
        KeyStr = x_d + ',' +y_d
        Reflect = RefInf['value']      
        Fit = self.countfitness(self.Frequency,Reflect)
        if KeyStr in DataDict.keys():
            if Fit > DataDict[KeyStr]:
               DataDict[KeyStr] = Fit
        else:
            DataDict[KeyStr] = Fit
    for xy in DataDict.keys():
        try:
            self.Xdata.append(float(xy.split(',')[0]))
            self.Ydata.append(float(xy.split(',')[1]))
            self.Fitness.append(DataDict[xy])
        except:
            pass
\end{lstlisting}

\mycontent{最后将得到的数据进行进行绘图展示,这里因为我们得到的数据是离散的,在绘制二维解空间图
时使用插值算法(12),最后使用了Matplotlib中contourf绘制等高线图(15)。}

\mycodelist{二维解空间绘图}

\setstretch{1.2}

\begin{lstlisting}
def DrawData(self):  
    if self.VarCheck():
        self.Xdata = list()
        self.Ydata = list()
        self.Fitness = list()
        self.DataFromDatabase()
        x = np.array(self.Xdata)
        y = np.array(self.Ydata)
        z = np.array(self.Fitness)
        xi = np.linspace(min(x), max(x), 1000)
        yi = np.linspace(min(y), max(y), 1000)
        zi = griddata(x,y,z,xi,yi,interp=u'linear')
        extent = (min(x),max(x),min(y),max(y))
        plt.figure(figsize=(10,8)) 
        plt.contourf(xi,yi,zi,100,linewidth=10.0,extent=extent,cmap ='jet')
        plt.colorbar().ax.tick_params(labelsize=24) 
        plt.tick_params(labelsize=30) 
        plt.xlabel(self.gridx.GetCellValue(0,0),fontsize=30) 
        plt.ylabel(self.gridy.GetCellValue(0,0),fontsize=30) 
        plt.tight_layout()
        plt.show()
    else:
        print '画图失败'  
\end{lstlisting}